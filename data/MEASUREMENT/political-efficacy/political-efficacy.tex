\documentclass[]{article}
\usepackage{lmodern}
\usepackage{amssymb,amsmath}
\usepackage{ifxetex,ifluatex}
\usepackage{fixltx2e} % provides \textsubscript
\ifnum 0\ifxetex 1\fi\ifluatex 1\fi=0 % if pdftex
  \usepackage[T1]{fontenc}
  \usepackage[utf8]{inputenc}
\else % if luatex or xelatex
  \ifxetex
    \usepackage{mathspec}
    \usepackage{xltxtra,xunicode}
  \else
    \usepackage{fontspec}
  \fi
  \defaultfontfeatures{Mapping=tex-text,Scale=MatchLowercase}
  \newcommand{\euro}{€}
\fi
% use upquote if available, for straight quotes in verbatim environments
\IfFileExists{upquote.sty}{\usepackage{upquote}}{}
% use microtype if available
\IfFileExists{microtype.sty}{%
\usepackage{microtype}
\UseMicrotypeSet[protrusion]{basicmath} % disable protrusion for tt fonts
}{}
\usepackage[margin=1in]{geometry}
\usepackage{longtable,booktabs}
\usepackage{graphicx}
\makeatletter
\def\maxwidth{\ifdim\Gin@nat@width>\linewidth\linewidth\else\Gin@nat@width\fi}
\def\maxheight{\ifdim\Gin@nat@height>\textheight\textheight\else\Gin@nat@height\fi}
\makeatother
% Scale images if necessary, so that they will not overflow the page
% margins by default, and it is still possible to overwrite the defaults
% using explicit options in \includegraphics[width, height, ...]{}
\setkeys{Gin}{width=\maxwidth,height=\maxheight,keepaspectratio}
\ifxetex
  \usepackage[setpagesize=false, % page size defined by xetex
              unicode=false, % unicode breaks when used with xetex
              xetex]{hyperref}
\else
  \usepackage[unicode=true]{hyperref}
\fi
\hypersetup{breaklinks=true,
            bookmarks=true,
            pdfauthor={},
            pdftitle={Political Efficacy in China and Mexico},
            colorlinks=true,
            citecolor=blue,
            urlcolor=blue,
            linkcolor=magenta,
            pdfborder={0 0 0}}
\urlstyle{same}  % don't use monospace font for urls
\setlength{\parindent}{0pt}
\setlength{\parskip}{6pt plus 2pt minus 1pt}
\setlength{\emergencystretch}{3em}  % prevent overfull lines
\setcounter{secnumdepth}{0}

%%% Use protect on footnotes to avoid problems with footnotes in titles
\let\rmarkdownfootnote\footnote%
\def\footnote{\protect\rmarkdownfootnote}

%%% Change title format to be more compact
\usepackage{titling}

% Create subtitle command for use in maketitle
\newcommand{\subtitle}[1]{
  \posttitle{
    \begin{center}\large#1\end{center}
    }
}

\setlength{\droptitle}{-2em}

  \title{Political Efficacy in China and Mexico}
    \pretitle{\vspace{\droptitle}\centering\huge}
  \posttitle{\par}
    \author{}
    \preauthor{}\postauthor{}
    \date{}
    \predate{}\postdate{}
  

\begin{document}

\maketitle


In 2002, the World Health Organization conducted a survey of two
provinces in China and three provinces in Mexico. This exercise is based
on:

\begin{quote}
Gary King, Christopher J. L. Murray, Joshua A. Salomon, and Ajay Tandon.
(2004). `\href{https://doi.org/10.1017/S000305540400108X}{Enhancing the
Validity and Cross-Cultural Comparability of Measurement in Survey
Research.}' \emph{American Political Science Review}, 98:1 (February),
pp.191-207.
\end{quote}

In this exercise we analyze respondents' views on their own political
efficacy. First, the following self-assessment question was asked.

\begin{quote}
How much say do you have in getting the government to address issues
that interest you?

\begin{enumerate}
\def\labelenumi{(\arabic{enumi})}
\setcounter{enumi}{4}
\itemsep1pt\parskip0pt\parsep0pt
\item
  Unlimited say, (4) A lot of say, (3) Some say, (2) Little say, (1) No
  say at all.
\end{enumerate}
\end{quote}

After the self-assessment question, three vignette questions were asked.

\begin{quote}
{[}Alison{]} lacks clean drinking water. She and her neighbors are
supporting an opposition candidate in the forthcoming elections that has
promised to address the issue. It appears that so many people in her
area feel the same way that the opposition candidate will defeat the
incumbent representative.

{[}Jane{]} lacks clean drinking water because the government is pursuing
an industrial development plan. In the campaign for an upcoming
election, an opposition party has promised to address the issue, but she
feels it would be futile to vote for the opposition since the government
is certain to win.

{[}Moses{]} lacks clean drinking water. He would like to change this,
but he can't vote, and feels that no one in the government cares about
this issue. So he suffers in silence, hoping something will be done in
the future.
\end{quote}

The respondent was asked to assess each vignette in the same manner as
the self-assessment question.

\begin{quote}
How much say does {[}`name'{]} in getting the government to address
issues that interest {[}him/her{]}?

\begin{enumerate}
\def\labelenumi{(\arabic{enumi})}
\setcounter{enumi}{4}
\itemsep1pt\parskip0pt\parsep0pt
\item
  Unlimited say, (4) A lot of say, (3) Some say, (2) Little say, (1) No
  say at all.
\end{enumerate}
\end{quote}

where {[}`name'{]} was replaced with either Alison, Jane, or Moses.

The data set we analyze \texttt{vignettes.csv} contains the following
variables:

\begin{longtable}[c]{@{}ll@{}}
\toprule\addlinespace
\begin{minipage}[b]{0.25\columnwidth}\raggedright
Name
\end{minipage} & \begin{minipage}[b]{0.68\columnwidth}\raggedright
Description
\end{minipage}
\\\addlinespace
\midrule\endhead
\begin{minipage}[t]{0.25\columnwidth}\raggedright
\texttt{self}
\end{minipage} & \begin{minipage}[t]{0.68\columnwidth}\raggedright
Self-assessment response
\end{minipage}
\\\addlinespace
\begin{minipage}[t]{0.25\columnwidth}\raggedright
\texttt{alison}
\end{minipage} & \begin{minipage}[t]{0.68\columnwidth}\raggedright
Response on Alison vignette
\end{minipage}
\\\addlinespace
\begin{minipage}[t]{0.25\columnwidth}\raggedright
\texttt{jane}
\end{minipage} & \begin{minipage}[t]{0.68\columnwidth}\raggedright
Response on Jane vignette
\end{minipage}
\\\addlinespace
\begin{minipage}[t]{0.25\columnwidth}\raggedright
\texttt{moses}
\end{minipage} & \begin{minipage}[t]{0.68\columnwidth}\raggedright
Response on Moses vignette
\end{minipage}
\\\addlinespace
\begin{minipage}[t]{0.25\columnwidth}\raggedright
\texttt{china}
\end{minipage} & \begin{minipage}[t]{0.68\columnwidth}\raggedright
1 for China and 0 for Mexico
\end{minipage}
\\\addlinespace
\begin{minipage}[t]{0.25\columnwidth}\raggedright
\texttt{age}
\end{minipage} & \begin{minipage}[t]{0.68\columnwidth}\raggedright
Age of respondent in years
\end{minipage}
\\\addlinespace
\bottomrule
\end{longtable}

In the analysis that follows, we assume that these survey responses can
be treated as numerical values. For example, \texttt{Unlimited say} = 5,
and \texttt{Little say} = 2.\\This approach is not appropriate if, for
example, the difference between \texttt{Unlimited say} and
\texttt{A lot of say} is not the same as the difference between
\texttt{Little say} and \texttt{No say at all}. However, relaxing this
assumption is beyond the scope of this chapter.

\subsection{Question 1}\label{question-1}

We begin by analyzing the self-assessment question. Plot the
distribution of responses separately for China and Mexico using
barplots, where the vertical axis is the proportion of respondents. In
addition, compute the mean response for each country. Which country's
respondents seemt to have a higher degree of political efficacy? Does
this seem plausible when Mexican citizens voted out out the
Institutional Revolutionary Party (PRI) after more than 70 years or rule
and Chinese citizens have not been able to vote in a fair election to
date?

\subsection{Question 2}\label{question-2}

We examine the possibility that any difference in the levels of efficacy
between Mexican and Chinese respondents is due to the difference in
their age distributions. Create histograms for the age variable
separately for Mexican and Chinese respondents. Add a vertical line
representing the median age of the respondents for each country. In
addition, use a Quantile-Quantile plot to compare the two age
distributions. What differences in age distribution do you observe
between the two countries? Answer this by interpreting each plot.

\subsection{Question 3}\label{question-3}

One problem of the self-assessment question is that survey respondents
may interpret the question differently. For example, two respondents who
choose the same response category may be facing quite different
political situations and hence may interpret \texttt{A lot of say}
differently. To address this problem, we rank a respondent's answer to
the self-assessment question relative to the same respondent's answer to
a vignette question. Compute the proportion of respondents, again
separately for China and Mexico, who ranks themselves (according to the
self-assessment question) as having less say in the government's
decisions than Moses (the last vignette). How does the result of this
analysis differ from that of the previous analysis? Give a brief
interpretation of the result.

\subsection{Question 4}\label{question-4}

Restrict the data to survey respondents who ranked these three vignettes
in the expected order (i.e., \texttt{Alison} $\ge$ \texttt{Jane} $\ge$
\texttt{Moses}). Now create a variable that represents how respondents
rank themselves relative to each vignette. This variable should be equal
to 1 if a respondent ranks themself lower than \texttt{Moses}, 2 if
ranked the same as \texttt{Moses} or higher than \texttt{Moses} but
lower than \texttt{Jane}, 3 if ranked the same as \texttt{Jane} or
higher than \texttt{Jane} but lower than \texttt{Alison}, and 4 if
ranked as high as \texttt{Alison} or higher. Create the barplots of this
new variable as in Question 1. The vertical axis should represent the
proportion of respondents for each response category. Also, compute the
mean value of this new variable separately for China and Mexico. Give a
brief interpretation of the result by comparing these results with those
obtained in Question 1.

\subsection{Question 5}\label{question-5}

Is the problem identified above more or less severe among older
respondents when compared to younger ones? Consider the previous
question for those who are 40 years or older and those who are younger
than 40 years. Do your conclusions between these two groups of
respondents? Relate your discussion to your finding for Question 2.

\end{document}
