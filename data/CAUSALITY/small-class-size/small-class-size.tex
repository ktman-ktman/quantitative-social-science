\documentclass[]{article}
\usepackage{lmodern}
\usepackage{amssymb,amsmath}
\usepackage{ifxetex,ifluatex}
\usepackage{fixltx2e} % provides \textsubscript
\ifnum 0\ifxetex 1\fi\ifluatex 1\fi=0 % if pdftex
  \usepackage[T1]{fontenc}
  \usepackage[utf8]{inputenc}
\else % if luatex or xelatex
  \ifxetex
    \usepackage{mathspec}
    \usepackage{xltxtra,xunicode}
  \else
    \usepackage{fontspec}
  \fi
  \defaultfontfeatures{Mapping=tex-text,Scale=MatchLowercase}
  \newcommand{\euro}{€}
\fi
% use upquote if available, for straight quotes in verbatim environments
\IfFileExists{upquote.sty}{\usepackage{upquote}}{}
% use microtype if available
\IfFileExists{microtype.sty}{%
\usepackage{microtype}
\UseMicrotypeSet[protrusion]{basicmath} % disable protrusion for tt fonts
}{}
\usepackage[margin=1in]{geometry}
\usepackage{longtable,booktabs}
\usepackage{graphicx}
\makeatletter
\def\maxwidth{\ifdim\Gin@nat@width>\linewidth\linewidth\else\Gin@nat@width\fi}
\def\maxheight{\ifdim\Gin@nat@height>\textheight\textheight\else\Gin@nat@height\fi}
\makeatother
% Scale images if necessary, so that they will not overflow the page
% margins by default, and it is still possible to overwrite the defaults
% using explicit options in \includegraphics[width, height, ...]{}
\setkeys{Gin}{width=\maxwidth,height=\maxheight,keepaspectratio}
\ifxetex
  \usepackage[setpagesize=false, % page size defined by xetex
              unicode=false, % unicode breaks when used with xetex
              xetex]{hyperref}
\else
  \usepackage[unicode=true]{hyperref}
\fi
\hypersetup{breaklinks=true,
            bookmarks=true,
            pdfauthor={},
            pdftitle={Efficacy of Small-class Size in Early Education},
            colorlinks=true,
            citecolor=blue,
            urlcolor=blue,
            linkcolor=magenta,
            pdfborder={0 0 0}}
\urlstyle{same}  % don't use monospace font for urls
\setlength{\parindent}{0pt}
\setlength{\parskip}{6pt plus 2pt minus 1pt}
\setlength{\emergencystretch}{3em}  % prevent overfull lines
\setcounter{secnumdepth}{0}

%%% Use protect on footnotes to avoid problems with footnotes in titles
\let\rmarkdownfootnote\footnote%
\def\footnote{\protect\rmarkdownfootnote}

%%% Change title format to be more compact
\usepackage{titling}

% Create subtitle command for use in maketitle
\newcommand{\subtitle}[1]{
  \posttitle{
    \begin{center}\large#1\end{center}
    }
}

\setlength{\droptitle}{-2em}

  \title{Efficacy of Small-class Size in Early Education}
    \pretitle{\vspace{\droptitle}\centering\huge}
  \posttitle{\par}
    \author{}
    \preauthor{}\postauthor{}
      \predate{\centering\large\emph}
  \postdate{\par}
    \date{5 August 2015}


\begin{document}

\maketitle


The STAR (Student-Teacher Achievement Ratio) Project is a four year
\emph{longitudinal study} examining the effect of class size in early
grade levels on educational performance and personal development.

This exercise is in part based on: Mosteller, Frederick. 1997.
``\href{http://dx.doi.org/10.2307/3824562}{The Tennessee Study of Class
Size in the Early School Grades.}'' \emph{Bulletin of the American
Academy of Arts and Sciences} 50(7): 14-25.

A longitudinal study is one in which the same participants are followed
over time. This particular study lasted from 1985 to 1989 involved
11,601 students. During the four years of the study, students were
randomly assigned to small classes, regular-sized classes, or
regular-sized classes with an aid. In all, the experiment cost around
\$12 million. Even though the program stopped in 1989 after the first
kindergarten class in the program finished third grade, collection of
various measurements (e.g., performance on tests in eighth grade,
overall high school GPA) continued through the end of participants' high
school attendance.

We will analyze just a portion of this data to investigate whether the
small class sizes improved performance or not. The data file name is
\texttt{STAR.csv}, which is a CSV data file. The names and descriptions
of variables in this data set are:

\begin{longtable}[c]{@{}ll@{}}
\toprule\addlinespace
\begin{minipage}[b]{0.25\columnwidth}\raggedright
Name
\end{minipage} & \begin{minipage}[b]{0.68\columnwidth}\raggedright
Description
\end{minipage}
\\\addlinespace
\midrule\endhead
\begin{minipage}[t]{0.25\columnwidth}\raggedright
\texttt{race}
\end{minipage} & \begin{minipage}[t]{0.68\columnwidth}\raggedright
Student's race (White = 1, Black = 2, Asian = 3, Hispanic = 4, Native
American = 5, Others = 6)
\end{minipage}
\\\addlinespace
\begin{minipage}[t]{0.25\columnwidth}\raggedright
\texttt{classtype}
\end{minipage} & \begin{minipage}[t]{0.68\columnwidth}\raggedright
Type of kindergarten class (small = 1, regular = 2, regular with aid =
3)
\end{minipage}
\\\addlinespace
\begin{minipage}[t]{0.25\columnwidth}\raggedright
\texttt{g4math}
\end{minipage} & \begin{minipage}[t]{0.68\columnwidth}\raggedright
Total scaled score for math portion of fourth grade standardized test
\end{minipage}
\\\addlinespace
\begin{minipage}[t]{0.25\columnwidth}\raggedright
\texttt{g4reading}
\end{minipage} & \begin{minipage}[t]{0.68\columnwidth}\raggedright
Total scaled score for reading portion of fourth grade standardized test
\end{minipage}
\\\addlinespace
\begin{minipage}[t]{0.25\columnwidth}\raggedright
\texttt{yearssmall}
\end{minipage} & \begin{minipage}[t]{0.68\columnwidth}\raggedright
Number of years in small classes
\end{minipage}
\\\addlinespace
\begin{minipage}[t]{0.25\columnwidth}\raggedright
\texttt{hsgrad}
\end{minipage} & \begin{minipage}[t]{0.68\columnwidth}\raggedright
High school graduation (did graduate = 1, did not graduate = 0)
\end{minipage}
\\\addlinespace
\bottomrule
\end{longtable}

Note that there are a fair amount of missing values in this data set.
For example, missing values arise because some students left a STAR
school before third grade or did not enter a STAR school until first
grade.

\subsection{Question 1}\label{question-1}

Create a new factor variable called \texttt{kinder} in the data frame.
This variable should recode \texttt{classtype} by changing integer
values to their corresponding informative labels (e.g., change 1 to
\texttt{small} etc.). Similarly, recode the \texttt{race} variable into
a factor variable with four levels (\texttt{white}, \texttt{black},
\texttt{hispanic}, \texttt{others}) by combining Asians and Native
Americans as the \texttt{others} category. For the \texttt{race}
variable, overwrite the original variable in the data frame rather than
creating a new one. Recall that \texttt{na.rm = TRUE} can be added to
functions in order to remove missing data.

\subsection{Question 2}\label{question-2}

How does performance on fourth grade reading and math tests for those
students assigned to a small class in kindergarten compare with those
assigned to a regular-sized class? Do students in the smaller classes
perform better? Use means to make this comparison while removing missing
values. Give a brief substantive interpretation of the results. To
understand the size of the estimated effects, compare them with the
standard deviation of the test scores.

\subsection{Question 3}\label{question-3}

Instead of comparing just average scores of reading and math tests
between those students assigned to small classes and those assigned to
regular-sized classes, look at the entire range of possible scores. To
do so, compare a high score, defined as the 66th percentile, and a low
score (the 33rd percentile) for small classes with the corresponding
score for regular classes. These are examples of \emph{quantile
treatment effects}. Does this analysis add anything to the analysis
based on mean in the previous question?

\subsection{Question 4}\label{question-4}

Some students were in small classes for all four years that the STAR
program ran. Others were assigned to small classes for only one year and
had either regular classes or regular classes with an aid for the rest.
How many such students of each type are in the data set? Create a
contingency table of proportions using the \texttt{kinder} and
\texttt{yearsmall} variables. Does participation in more years of small
classes make a greater difference in test scores? Compare the average
and median reading and math test scores across students who spent
different numbers of years in small classes.

\subsection{Question 5}\label{question-5}

We examine whether the STAR program reduced the achievement gaps across
different racial groups. Begin by comparing the average reading and math
test scores between white and minority students (i.e., Blacks and
hispanics) among those students who were assigned to regular classes
with no aid. Conduct the same comparison among those students who were
assigned to small classes. Give a brief substantive interpretation of
the results of your analysis.

\subsection{Question 6}\label{question-6}

We consider the long term effects of kindergarden class size. Compare
high school graduation rates across students assigned to different class
types. Also, examine whether graduation rates differ by the number of
years spent in small classses. Finally, as done in the previous
question, investigate whether the STAR program has reduced the racial
gap between white and minority students' graduation rates. Briefly
discuss the results.

\end{document}
